% !TEX root = MA.tex
\section{Vorwort}
\label{sec:vorwort}

Wer es schon einmal ausprobiert hat weiss es, selber angepflanztes Gemüse hat einen viel intensieveren Geschmack, als das Gemüse, welches in Supermärkten wie Coop oder Migros zu kaufen ist. Daher haben wir als Familie schon seit ich denken kann einen eigenen Gemüsegarten. Darin pflanzen wir alles von Zwiebeln und Kartoffeln bis zum Kopfsalat selber an. Jedoch ist es schon öfters vorgekommen, dass die Ernte sehr klein war oder sogar ausfiel, weil das Wetter nicht mitspielte. Ein weiterer Grund für misslungene Ernten war ich, denn den grünen Daumen meiner Eltern, habe ich definitiv nicht geerbt.
Daher kam mir die Idee ein wetterunabhängiges autonomes Indoorgewächshaus zu bauen und programmieren, da es alle negativen Faktoren eliminiert. Das Indoorgewächshaus sollte selbstständig den Tag-Nacht Rhytmus kontrollieren, die Pflanzen giessen und die Temperatur im Indoorgewächshaus kontrollieren. \newline Als wir uns für ein Thema für die Maturaarbeit entscheiden mussten errinnerte ich mich an die Idee welche ich im vorherigen Herbst hatte und entschied mich dafür meine Maturaarbeit darüber zu schreiben. Nach etwas etwas überlegung konnte ich die Fragestellung wie folgt definieren:
\begin{itemize}
  \item \maFRAGE
\end{itemize}
